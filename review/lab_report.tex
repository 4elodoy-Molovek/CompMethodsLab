% ===================================================================
% ОТЧЁТ ПО ЛАБОРАТОРНОЙ РАБОТЕ
% "Оптимизация последовательности переработки партий сахарной свёклы"
% ===================================================================
%
% НАЗНАЧЕНИЕ:
%   Этот файл содержит полный отчёт по лабораторной работе в формате LaTeX.
%   Отчёт включает постановку задачи, описание работы программы и список литературы.
%
% КАК КОМПИЛИРОВАТЬ:
%   1. Установите LaTeX дистрибутив (MiKTeX для Windows, TeX Live для Linux/Mac)
%   2. Откройте терминал в папке с этим файлом
%   3. Выполните команду: pdflatex lab_report.tex
%   4. Повторите команду ещё раз (для правильной нумерации ссылок)
%   5. Результат: lab_report.pdf
%
% СТРУКТУРА ДОКУМЕНТА:
%   - Титульная страница (заполните данные студента и руководителя)
%   - Содержание (автоматически генерируется)
%   - Постановка задачи (математическая модель)
%   - Как работает программа (описание системы)
%   - Заключение
%   - Список литературы
%
% КАК РЕДАКТИРОВАТЬ:
%   1. Титульная страница: найдите секцию "Титульная страница" и замените
%      [номер группы], [Фамилия Имя Отчество], [Дата] на реальные данные
%   2. Для добавления новых разделов используйте: \section{Название}
%   3. Для подразделов: \subsection{Название}
%   4. Для формул используйте окружения: $...$ (в строке) или \[...\] (отдельно)
%   5. Для списков используйте: \begin{itemize} ... \end{itemize}
%
% МАТЕМАТИЧЕСКИЕ ФОРМУЛЫ:
%   - В строке: $x = y + z$
%   - Отдельно: \[ x = y + z \]
%   - Нумерованные: \begin{equation} ... \end{equation}
%   - Системы: \begin{align} ... \end{align}
%
% СПИСОК ЛИТЕРАТУРЫ:
%   - Добавьте новые источники в секцию "Список литературы"
%   - Используйте формат: \item Автор. Название / Автор. — Изд. — Год. — Стр.
%   - Для ссылок на веб-ресурсы используйте: \url{https://...}
%
% ВАЖНО:
%   - Сохраняйте файл в кодировке UTF-8
%   - После изменений перекомпилируйте документ
%   - Проверяйте результат в PDF перед отправкой
%
% АВТОР: [Ваше имя]
% ДАТА СОЗДАНИЯ: [Дата]
% ===================================================================

\documentclass[a4paper,12pt]{article}
% Класс документа: article (статья)
% Размер бумаги: A4
% Размер шрифта: 12pt (стандартный для научных работ)

% ===================================================================
% ПОДКЛЮЧЕНИЕ ПАКЕТОВ
% ===================================================================
% Пакеты расширяют возможности LaTeX. Каждый пакет добавляет
% определённую функциональность (математика, таблицы, ссылки и т.д.)
% ===================================================================

% --- Подключение пакетов ---

% Кодировка входного файла (UTF-8 поддерживает русские буквы)
\usepackage[utf8]{inputenc}

% Кодировка шрифтов для вывода (T2A - кириллица)
\usepackage[T2A]{fontenc}

% Язык документа (русский) - для правильной типографики и переносов
\usepackage[russian]{babel}
% Математические пакеты:
%   amsmath - расширенная математика (формулы, выравнивание)
%   amssymb - дополнительные математические символы
%   amsthm  - теоремы, леммы, определения
\usepackage{amsmath, amssymb, amsthm}

% Настройка полей страницы
\usepackage{geometry}

% Цвета (для оформления)
\usepackage{xcolor}

% Выделение кода (если нужно вставить примеры кода)
\usepackage{listings}

% Красивые цветные рамки (для выделения важных блоков)
\usepackage[most]{tcolorbox}

% Гиперссылки (автоматически делает ссылки в PDF кликабельными)
\usepackage{hyperref}

% Расширенные списки (настройка маркеров, отступов)
\usepackage{enumitem}

% Профессиональные таблицы (улучшенное оформление)
\usepackage{booktabs}

% Управление размещением рисунков
\usepackage{float}

% Рисование схем и диаграмм (если нужно добавить графики)
\usepackage{tikz}
% Библиотеки TikZ для различных элементов
\usetikzlibrary{positioning, shapes, arrows.meta, calc, trees, decorations.pathreplacing, matrix}

% ===================================================================
% НАСТРОЙКА ДОКУМЕНТА
% ===================================================================

% --- Настройка полей страницы ---
% Стандартные поля для научных работ (ГОСТ)
\geometry{top=1.5cm, bottom=1.5cm, left=2cm, right=2cm}
% Если нужно изменить поля, измените значения выше

% --- Отключение нумерации разделов ---
% secnumdepth=0 означает, что разделы не будут нумероваться
% (например, будет просто "Постановка задачи" вместо "1. Постановка задачи")
% Если нужна нумерация, закомментируйте эту строку или измените на 2
\setcounter{secnumdepth}{0}

% --- Настройка ссылок ---
% Делает ссылки цветными и кликабельными в PDF
% linkcolor - цвет внутренних ссылок (на разделы, формулы)
% urlcolor  - цвет внешних ссылок (URL)
\hypersetup{colorlinks=true, linkcolor=blue!60!black, urlcolor=blue!60!black}

% ===================================================================
% НАСТРОЙКА ЦВЕТОВ И СТИЛЕЙ
% ===================================================================
% Эти цвета используются для оформления кода (если будете вставлять примеры)
% Можно изменить цвета по своему вкусу
% --- Цвета и стиль кода ---
\definecolor{codegreen}{rgb}{0,0.5,0}      % Зелёный для комментариев
\definecolor{codegray}{rgb}{0.5,0.5,0.5}  % Серый для строк
\definecolor{codeblue}{rgb}{0.1,0.1,0.8}  % Синий для ключевых слов
\definecolor{backcolour}{rgb}{0.97,0.97,0.97} % Светло-серый фон для кода

% ===================================================================
% НАСТРОЙКА КИРИЛЛИЦЫ В КОДЕ
% ===================================================================
% Эта секция настраивает поддержку русских букв в блоках кода.
% Каждая буква маппится на соответствующий символ в кодировке T2A.
% Это нужно для правильного отображения комментариев на русском языке
% в примерах кода (если будете их вставлять).
% ===================================================================
% --- НАСТРОЙКА КИРИЛЛИЦЫ В КОДЕ ---
\lstset{
    inputencoding=utf8,      % Кодировка входного файла
    extendedchars=true,      % Расширенные символы (кириллица)
    literate={а}{{\selectfont\char224}}1
             {б}{{\selectfont\char225}}1
             {в}{{\selectfont\char226}}1
             {г}{{\selectfont\char227}}1
             {д}{{\selectfont\char228}}1
             {е}{{\selectfont\char229}}1
             {ё}{{\"e}}1
             {ж}{{\selectfont\char230}}1
             {з}{{\selectfont\char231}}1
             {и}{{\selectfont\char232}}1
             {й}{{\selectfont\char233}}1
             {к}{{\selectfont\char234}}1
             {л}{{\selectfont\char235}}1
             {м}{{\selectfont\char236}}1
             {н}{{\selectfont\char237}}1
             {о}{{\selectfont\char238}}1
             {п}{{\selectfont\char239}}1
             {р}{{\selectfont\char240}}1
             {с}{{\selectfont\char241}}1
             {т}{{\selectfont\char242}}1
             {у}{{\selectfont\char243}}1
             {ф}{{\selectfont\char244}}1
             {х}{{\selectfont\char245}}1
             {ц}{{\selectfont\char246}}1
             {ч}{{\selectfont\char247}}1
             {ш}{{\selectfont\char248}}1
             {щ}{{\selectfont\char249}}1
             {ъ}{{\selectfont\char250}}1
             {ы}{{\selectfont\char251}}1
             {ь}{{\selectfont\char252}}1
             {э}{{\selectfont\char253}}1
             {ю}{{\selectfont\char254}}1
             {я}{{\selectfont\char255}}1
             {А}{{\selectfont\char192}}1
             {Б}{{\selectfont\char193}}1
             {В}{{\selectfont\char194}}1
             {Г}{{\selectfont\char195}}1
             {Д}{{\selectfont\char196}}1
             {Е}{{\selectfont\char197}}1
             {Ё}{{\"E}}1
             {Ж}{{\selectfont\char198}}1
             {З}{{\selectfont\char199}}1
             {И}{{\selectfont\char200}}1
             {Й}{{\selectfont\char201}}1
             {К}{{\selectfont\char202}}1
             {Л}{{\selectfont\char203}}1
             {М}{{\selectfont\char204}}1
             {Н}{{\selectfont\char205}}1
             {О}{{\selectfont\char206}}1
             {П}{{\selectfont\char207}}1
             {Р}{{\selectfont\char208}}1
             {С}{{\selectfont\char209}}1
             {Т}{{\selectfont\char210}}1
             {У}{{\selectfont\char211}}1
             {Ф}{{\selectfont\char212}}1
             {Х}{{\selectfont\char213}}1
             {Ц}{{\selectfont\char214}}1
             {Ч}{{\selectfont\char215}}1
             {Ш}{{\selectfont\char216}}1
             {Щ}{{\selectfont\char217}}1
             {Ъ}{{\selectfont\char218}}1
             {Ы}{{\selectfont\char219}}1
             {Ь}{{\selectfont\char220}}1
             {Э}{{\selectfont\char221}}1
             {Ю}{{\selectfont\char222}}1
             {Я}{{\selectfont\char223}}1
}

% ===================================================================
% СТИЛЬ ДЛЯ ВЫДЕЛЕНИЯ КОДА
% ===================================================================
% Настройка внешнего вида блоков с кодом (если будете вставлять примеры).
% Можно изменить цвета, размер шрифта, отступы и т.д.
% ===================================================================
\lstdefinestyle{mystyle}{
    backgroundcolor=\color{backcolour},   % Цвет фона (светло-серый)
    commentstyle=\color{codegreen}\itshape, % Стиль комментариев (зелёный, курсив)
    keywordstyle=\color{codeblue}\bfseries,  % Стиль ключевых слов (синий, жирный)
    numberstyle=\tiny\color{codegray},      % Стиль номеров строк (маленький, серый)
    stringstyle=\color{codegray},           % Стиль строк (серый)
    basicstyle=\ttfamily\small,             % Базовый стиль (моноширинный, маленький)
    breakatwhitespace=false,                % Не переносить в середине слова
    breaklines=true,                        % Переносить длинные строки
    captionpos=b,                           % Позиция подписи (bottom - снизу)
    keepspaces=true,                        % Сохранять пробелы
    numbers=left,                           % Номера строк слева
    numbersep=5pt,                         % Отступ номеров от кода
    showspaces=false,                      % Не показывать пробелы специальными символами
    showstringspaces=false,                % Не показывать пробелы в строках
    showtabs=false,                        % Не показывать табуляции
    tabsize=4,                             % Размер табуляции (4 пробела)
    aboveskip=10pt,                        % Отступ сверху
    belowskip=10pt,                        % Отступ снизу
    language=Python,                       % Язык программирования (для подсветки синтаксиса)
    % Дополнительные ключевые слова Python (для правильной подсветки)
    morekeywords={def, class, import, from, return, if, else, elif, for, while, try, except, True, False, None, lambda, self},
    mathescape=true                        % Разрешить математические формулы в коде
}

% ===================================================================
% НАСТРОЙКА ЦВЕТНЫХ БЛОКОВ
% ===================================================================
% Эти команды создают красивые цветные рамки для выделения важных блоков.
% Можно использовать для определений, лемм, теорем и т.д.
%
% ИСПОЛЬЗОВАНИЕ:
%   \begin{definitionbox}{Определение}
%       Текст определения...
%   \end{definitionbox}
%
% Доступные блоки:
%   - definitionbox - зелёная рамка (для определений)
%   - lemmabox      - синяя рамка (для лемм)
%   - theorembox    - красная рамка (для теорем)
% ===================================================================
% --- Блоки ---
\newtcolorbox{definitionbox}[1]{colback=green!5!white, colframe=green!40!black, fonttitle=\bfseries, title=#1, sharp corners=downhill}
\newtcolorbox{lemmabox}[1]{colback=blue!5!white, colframe=blue!40!black, fonttitle=\bfseries, title=#1}
\newtcolorbox{theorembox}[1]{colback=red!5!white, colframe=red!40!black, fonttitle=\bfseries, title=#1}

\begin{document}

% ===================================================================
% ТИТУЛЬНАЯ СТРАНИЦА
% ===================================================================
% ВАЖНО: Заполните все поля в квадратных скобках [ ] своими данными!
% ===================================================================
\begin{titlepage}
\centering

% Название университета (не менять, если не требуется)
{\Large Национальный исследовательский Нижегородский государственный университет имени Н. И. Лобачевского}

\vspace{0.5cm}

% Институт (не менять, если не требуется)
(Институт информационных технологий, математики и механики)

\vspace{1cm}

% Кафедра (не менять, если не требуется)
{\large Кафедра: Алгебра, геометрия и дискретная математика (АГДМ)}

\vspace{2cm}

% Тип работы
{\Large \textbf{Отчёт по лабораторной работе}}

\vspace{1cm}

% Название работы (можно изменить, если нужно)
{\large «Оптимизация последовательности переработки партий сахарной свёклы»}

\vspace{3cm}

% Информация о студенте и руководителе
% ВАЖНО: Замените [номер группы] на реальный номер вашей группы
% ВАЖНО: Замените [Фамилия Имя Отчество] на ваше ФИО
\begin{flushright}
Выполнил:\\
Студент группы [номер группы]\\
[Фамилия Имя Отчество]

\vspace{1cm}

Научный руководитель:\\
[Фамилия Имя Отчество]
\end{flushright}

\vspace{2cm}

% Дата (замените [Дата] на реальную дату, например: 15 декабря 2024)
\centering
[Дата] г.

\end{titlepage}

\newpage

% ===================================================================
% СОДЕРЖАНИЕ
% ===================================================================
% Автоматически генерируется на основе \section и \subsection
% Если добавите новые разделы, они автоматически появятся здесь
% ===================================================================
\tableofcontents
\newpage

% ===================================================================
% ОСНОВНОЕ СОДЕРЖАНИЕ ОТЧЁТА
% ===================================================================
% Ниже идут основные разделы отчёта.
% Каждый раздел начинается с команды \section{Название}
% Подразделы создаются командой \subsection{Название}
% ===================================================================

\section{Постановка задачи}
% Этот раздел содержит математическую постановку задачи.
% Описывает модель деградации свёклы, формулы, параметры.
% Если нужно добавить новые подразделы, используйте \subsection{}

\subsection{Математическая модель деградации свёклы}

Рассматривается задача оптимизации последовательности переработки $n$ партий сахарной свёклы на сахарном заводе. За единицу времени завод перерабатывает партию сырья массой $M$ одного сорта.

Пусть $i = \overline{1, n}$ — номер партии, $j = \overline{1, n}$ — этап переработки. Обозначим через $c_{ij}$ количество полезного ингредиента (сахара) в начале этапа $j$ для партии $i$.

Параметры $c_{ij}$ определяются через начальную долю полезного ингредиента $c_{i1} = a_i \in [a_{\min}, a_{\max}]$ и коэффициенты деградации $b_{i,j-1} = \frac{c_{ij}}{c_{i,j-1}}$ для $j = 2, \dots, n$. Соответствующие формулы имеют вид:

\begin{align}
c_{i1} &= a_i, \\
c_{ij} &= a_i b_{i1} \dots b_{i,j-1}, \quad i = 1, \dots, n, \quad j = 2, \dots, n.
\end{align}

\subsection{Процесс дозаривания свёклы}

Опционально учитывается процесс дозаривания (дозревания) свёклы:
\begin{itemize}
    \item Число этапов дозаривания ограничено и равно $v$, где $2 \leq v \leq \left\lfloor \frac{n}{2} \right\rfloor$.
    \item На этапах дозаривания параметры $b_{ij} \in (1, \beta_{\text{MAX}}]$ для $j = \overline{1, v - 1}$.
    \item На этапах увядания параметры $b_{ij} \in [\beta_1, \beta_2] \subset (0,1)$ для $j = \overline{v, n - 1}$.
    \item Рекомендуется брать $\beta_{\text{MAX}} = \frac{n - 1}{n - 2}$.
\end{itemize}

\subsection{Распределение коэффициентов}

Коэффициенты деградации $b_{ij}$ могут распределяться двумя способами:

\begin{enumerate}
    \item \textbf{Равномерное распределение:} $b_{ij}$ распределены равномерно на отрезке $[\beta_1, \beta_2]$ (для увядания) или $(1, \beta_{\text{MAX}}]$ (для дозаривания).
    
    \item \textbf{Концентрированное распределение:} Для каждой партии $i$ существует константа $\delta_i \leq \left| \frac{\beta_2 - \beta_1}{4} \right|$, такая что $b_{ij} \in [\beta_i^1, \beta_i^2] \subset [\beta_1, \beta_2]$, где $|\beta_i^1 - \beta_i^2| = \delta_i$. На этом подотрезке параметры распределены равномерно.
\end{enumerate}

\subsection{Матрица состояний и потери}

Имеется матрица состояний $\widetilde{S} = C - \widetilde{L}$, где:
\begin{itemize}
    \item $C$ — матрица коэффициентов содержания сахара для каждой партии (строки) на каждом этапе (столбцы).
    \item $\widetilde{L}$ — матрица с неотрицательными элементами потерь сахара.
    \item $\widetilde{S}$ — матрица коэффициентов содержания сахара после учёта потерь.
\end{itemize}

Элемент $l_{ij}$ матрицы потерь (в \%) вычисляется по формуле:
\begin{equation}
l_{ij} = 1.1 + 0.1541 \times (K_i + Na_i) + 0.2159 \times N_i + 0.9989 \times I_{ij} + 0.1967,
\end{equation}
где:
\begin{itemize}
    \item $K_i \in [4.8; 7.05]$ — содержание калия;
    \item $Na_i \in [0.21; 0.82]$ — содержание натрия;
    \item $N_i \in [1.58; 2.8]$ — содержание азота;
    \item $I_{ij} = I_{i0} \times (1.029)^{7j - 7}$, где $I_{i0} \in [0.62; 0.64]$ — начальный индекс, $j$ — номер этапа (нумерация с 1), 7 — количество дней в этапе.
\end{itemize}

Потери учитываются при расчёте итоговой матрицы: $\widetilde{S} = C - \frac{L}{100}$.

\subsection{Задача оптимизации}

Последовательность переработки сырья описывается перестановкой $\sigma$ натуральных чисел от 1 до $n$:

$$\sigma = \begin{pmatrix} 
1 & 2 & \dots & n-1 & n \\ 
\sigma(1) & \sigma(2) & \dots & \sigma(n-1) & \sigma(n) 
\end{pmatrix}$$

Общий выход сахара рассчитывается по формуле:
\begin{equation}
S(\sigma) = \widetilde{s}_{\sigma(1),1} + \widetilde{s}_{\sigma(2),2} + \dots + \widetilde{s}_{\sigma(n),n}
\end{equation}

Масса конечного продукта вычисляется как:
\begin{equation}
S = S(\sigma) \times M \times d,
\end{equation}
где $d = 7$ — число дней в этапе.

\textbf{Цель оптимизации:} найти такую перестановку $\sigma$ чисел от 1 до $n$ (последовательность переработки имеющихся партий сырья), при которой $S(\sigma)$ будет максимальным.

\subsection{Особенности задачи}

В начале этапа $j$ известны только первые $j$ столбцов матрицы, причём только для партий, ещё не переработанных. Это делает задачу нечёткой задачей о назначениях, что исключает применение точных алгоритмов (например, венгерского алгоритма) в реальном времени. Однако в рамках компьютерного эксперимента возможно сравнение различных эвристических стратегий с оптимальным решением, полученным венгерским алгоритмом.

\section{Как работает программа}
% Этот раздел описывает архитектуру и работу разработанной системы.
% Включает описание компонентов, алгоритмов, интерфейса.
% Можно добавить схемы, используя TikZ (см. примеры в интернете)

Разработанная система представляет собой веб-приложение на базе Electron с Python backend, предназначенное для моделирования и оптимизации процесса переработки сахарной свёклы.

\subsection{Архитектура системы}

Система состоит из трёх основных компонентов:

\begin{enumerate}
    \item \textbf{Frontend (Electron):} Графический интерфейс пользователя, реализованный на HTML/CSS/JavaScript с использованием Electron для создания десктопного приложения.
    
    \item \textbf{Backend (Flask):} REST API сервер на Python, обрабатывающий запросы на симуляцию и оптимизацию.
    
    \item \textbf{Модули обработки данных:} Генерация матриц состояний, расчёт потерь, применение алгоритмов оптимизации.
\end{enumerate}

\subsection{Процесс работы программы}

\subsubsection{Генерация данных}

При запуске симуляции система выполняет следующие шаги:

\begin{enumerate}
    \item \textbf{Генерация партий свёклы:} Для каждой партии $i = 1, \dots, n$ генерируются случайные параметры:
    \begin{itemize}
        \item Начальная сахаристость $a_i \in [a_{\min}, a_{\max}]$;
        \item Содержание калия $K_i \in [4.8; 7.05]$;
        \item Содержание натрия $Na_i \in [0.21; 0.82]$;
        \item Содержание азота $N_i \in [1.58; 2.8]$;
        \item Начальный индекс $I_{i0} \in [0.62; 0.64]$.
    \end{itemize}
    
    \item \textbf{Генерация коэффициентов деградации:} Формируется матрица $B$ размером $n \times n$, где элемент $B[i, j]$ — коэффициент перехода от этапа $j-1$ к этапу $j$ для партии $i$:
    \begin{itemize}
        \item Если включено дозаривание и $j \leq v-1$: $b_{ij} \in (1, \beta_{\text{MAX}}]$;
        \item Иначе (увядание): $b_{ij} \in [\beta_1, \beta_2] \subset (0,1)$;
        \item Распределение может быть равномерным или концентрированным.
    \end{itemize}
    
    \item \textbf{Расчёт матрицы состояний $C$:} Вычисляется матрица содержания сахара:
    \begin{align}
    C[i, 0] &= a_i, \\
    C[i, j] &= C[i, j-1] \times B[i, j], \quad j = 1, \dots, n-1.
    \end{align}
    
    \item \textbf{Расчёт матрицы потерь $L$:} Для каждого элемента вычисляется:
    \begin{equation}
    L[i, j] = 1.1 + 0.1541 \times (K_i + Na_i) + 0.2159 \times N_i + 0.9989 \times I_{ij} + 0.1967,
    \end{equation}
    где $I_{ij} = I_{i0} \times (1.029)^{7(j-1)}$.
    
    \item \textbf{Формирование итоговой матрицы $\widetilde{S}$:} Вычисляется матрица после учёта потерь:
    \begin{equation}
    \widetilde{S} = C - \frac{L}{100}.
    \end{equation}
\end{enumerate}

\subsubsection{Алгоритмы оптимизации}

Система реализует несколько стратегий оптимизации последовательности переработки:

\begin{enumerate}
    \item \textbf{Жадная стратегия (Greedy):} На каждом этапе $j$ выбирается доступная партия $i$ с максимальным значением $\widetilde{S}[i, j]$.
    
    \item \textbf{Бережливая стратегия (Thrifty):} На каждом этапе $j$ выбирается доступная партия $i$ с минимальным значением $\widetilde{S}[i, j]$.
    
    \item \textbf{Бережливая/Жадная (Thrifty/Greedy):} Первые $(n-\nu)$ этапов используют бережливую стратегию, затем с этапа $\nu$ применяется жадная стратегия. По умолчанию $\nu = \left\lfloor \frac{n}{2} \right\rfloor$.
    
    \item \textbf{Жадная/Бережливая (Greedy/Thrifty):} Первые $(n-\nu)$ этапов используют жадную стратегию, затем с этапа $\nu$ применяется бережливая стратегия.
    
    \item \textbf{Стратегия T(k)G:} Первые $(n-\nu)$ этапов выбирают $k$-ю позицию из отсортированного по возрастанию списка значений сахаристости, затем применяется жадная стратегия.
    
    \item \textbf{Стратегия Gk:} Вариация жадной стратегии с измерениями перед первым этапом и после этапов, кратных $k$. Выбираются $k$ партий с наивысшей сахаристостью и обрабатываются в порядке убывания.
    
    \item \textbf{Венгерский алгоритм:} Точный алгоритм решения задачи о назначениях, дающий оптимальное решение $S^*$ для сравнения с эвристиками.
\end{enumerate}

\subsubsection{Расчёт результатов}

Для каждой стратегии вычисляются:
\begin{itemize}
    \item Перестановка $\sigma$ — последовательность номеров партий;
    \item Выход сахара $S(\sigma) = \sum_{j=1}^{n} \widetilde{s}_{\sigma(j), j}$;
    \item Масса конечного продукта $S = S(\sigma) \times M \times 7$.
\end{itemize}

\subsubsection{Интерфейс пользователя}

Графический интерфейс позволяет:
\begin{itemize}
    \item Задавать параметры эксперимента (количество партий, диапазоны значений, тип распределения);
    \item Включать/выключать дозаривание и потери;
    \item Запускать симуляцию и просматривать результаты;
    \item Сравнивать эффективность различных стратегий;
    \item Визуализировать матрицы состояний и результаты оптимизации.
\end{itemize}

\subsubsection{Валидация входных данных}

Система выполняет проверку корректности входных параметров:
\begin{itemize}
    \item Проверка положительности $n$ и $M$;
    \item Проверка корректности диапазонов $[a_{\min}, a_{\max}]$ и $[\beta_1, \beta_2]$;
    \item Проверка условий для дозаривания: $2 \leq v \leq \left\lfloor \frac{n}{2} \right\rfloor$;
    \item Проверка типов распределения и других параметров.
\end{itemize}

\section{Заключение}
% Краткое резюме работы, основные результаты, выводы.
% Можно добавить перспективы развития проекта.

В рамках лабораторной работы была разработана система поддержки принятия решений для оптимизации последовательности переработки партий сахарной свёклы. Система позволяет моделировать процесс деградации свёклы с учётом дозаривания и потерь, сравнивать различные эвристические стратегии оптимизации и находить оптимальное решение с помощью венгерского алгоритма.

Реализованные алгоритмы демонстрируют различные подходы к решению задачи: от простых жадных стратегий до комбинированных методов, учитывающих особенности процесса переработки. Сравнение результатов различных стратегий позволяет выбрать наиболее эффективный подход для конкретных условий производства.

\section{Список литературы}
% ===================================================================
% СПИСОК ЛИТЕРАТУРЫ
% ===================================================================
% Формат оформления источников:
%   - Книги: Автор. Название / Автор. — Издание. — Город: Издательство, Год. — Страницы.
%   - Статьи: Автор. Название / Автор // Журнал. — Год. — Том, Номер. — Страницы.
%   - Веб-ресурсы: Автор. Название [Электронный ресурс]. — Режим доступа: \url{URL} (дата обращения: Год).
%
% КАК ДОБАВИТЬ НОВЫЙ ИСТОЧНИК:
%   1. Скопируйте формат одного из существующих источников
%   2. Замените данные на свои
%   3. Добавьте \item перед новым источником
%   4. После изменений перекомпилируйте документ
% ===================================================================

\begin{enumerate}
    % Книга: классический учебник по алгоритмам
    \item Кнут Д. Э. Искусство программирования. Том 3. Сортировка и поиск / Д. Э. Кнут. — 2-е изд. — М.: Вильямс, 2000. — 832 с.
    
    % Книга: современный учебник по алгоритмам (используется в курсе)
    \item Кормен Т., Лейзерсон Ч., Ривест Р., Штайн К. Алгоритмы: построение и анализ / Т. Кормен [и др.]. — 3-е изд. — М.: Вильямс, 2013. — 1328 с.
    
    % Статья: оригинальная работа по венгерскому алгоритму
    \item Kuhn H. W. The Hungarian method for the assignment problem / H. W. Kuhn // Naval Research Logistics Quarterly. — 1955. — Vol. 2, No. 1-2. — P. 83-97.
    
    % Статья: улучшение венгерского алгоритма
    \item Munkres J. Algorithms for the Assignment and Transportation Problems / J. Munkres // Journal of the Society for Industrial and Applied Mathematics. — 1957. — Vol. 5, No. 1. — P. 32-38.
    
    % Веб-ресурс: документация библиотеки SciPy (используется в проекте)
    \item SciPy Community. scipy.optimize.linear\_sum\_assignment — SciPy v1.11.0 Manual [Электронный ресурс]. — Режим доступа: \url{https://docs.scipy.org/doc/scipy/reference/generated/scipy.optimize.linear_sum_assignment.html} (дата обращения: 2024).
    
    % Веб-ресурс: документация Flask (используется для backend)
    \item Flask Development Team. Flask — Web Development, One Drop at a Time [Электронный ресурс]. — Режим доступа: \url{https://flask.palletsprojects.com/} (дата обращения: 2024).
    
    % Веб-ресурс: документация Electron (используется для desktop приложения)
    \item Electron Contributors. Electron — Build cross-platform desktop apps with JavaScript, HTML, and CSS [Электронный ресурс]. — Режим доступа: \url{https://www.electronjs.org/} (дата обращения: 2024).
    
    % Веб-ресурс: документация NumPy (используется для вычислений)
    \item NumPy Developers. NumPy — The fundamental package for scientific computing with Python [Электронный ресурс]. — Режим доступа: \url{https://numpy.org/} (дата обращения: 2024).
\end{enumerate}

% ===================================================================
% КОНЕЦ ДОКУМЕНТА
% ===================================================================
% После этой строки ничего не добавляйте!
% ===================================================================

\end{document}

