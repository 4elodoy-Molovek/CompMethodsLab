% ===================================================================
% МАТЕРИАЛЫ ДЛЯ ЗАЩИТЫ ЛАБОРАТОРНОЙ РАБОТЫ
% "Оптимизация последовательности переработки партий сахарной свёклы"
% ===================================================================

\documentclass[a4paper,12pt]{article}

% ===================================================================
% ПОДКЛЮЧЕНИЕ ПАКЕТОВ (Аналогично lab_report.tex)
% ===================================================================

\usepackage[utf8]{inputenc}
\usepackage[T2A]{fontenc}
\usepackage[russian]{babel}
\usepackage{amsmath, amssymb, amsthm}
\usepackage{geometry}
\usepackage{xcolor}
\usepackage{listings}
\usepackage[most]{tcolorbox}
\usepackage{hyperref}
\usepackage{enumitem}
\usepackage{booktabs}
\usepackage{float}
\usepackage{tikz}
\usetikzlibrary{positioning, shapes, arrows.meta, calc, trees, decorations.pathreplacing, matrix}

% ===================================================================
% НАСТРОЙКА ДОКУМЕНТА
% ===================================================================

\geometry{top=1.5cm, bottom=1.5cm, left=2cm, right=2cm}
\setcounter{secnumdepth}{0} % Отключаем нумерацию разделов для удобства чтения сценария
\hypersetup{colorlinks=true, linkcolor=blue!60!black, urlcolor=blue!60!black}

% Цвета
\definecolor{codegreen}{rgb}{0,0.5,0}
\definecolor{codegray}{rgb}{0.5,0.5,0.5}
\definecolor{codeblue}{rgb}{0.1,0.1,0.8}
\definecolor{backcolour}{rgb}{0.97,0.97,0.97}

% Настройка листинга (для сохранения совместимости)
\lstset{
    inputencoding=utf8,
    extendedchars=true,
    literate={а}{{\selectfont\char224}}1
             {б}{{\selectfont\char225}}1
             {в}{{\selectfont\char226}}1
             {г}{{\selectfont\char227}}1
             {д}{{\selectfont\char228}}1
             {е}{{\selectfont\char229}}1
             {ё}{{\"e}}1
             {ж}{{\selectfont\char230}}1
             {з}{{\selectfont\char231}}1
             {и}{{\selectfont\char232}}1
             {й}{{\selectfont\char233}}1
             {к}{{\selectfont\char234}}1
             {л}{{\selectfont\char235}}1
             {м}{{\selectfont\char236}}1
             {н}{{\selectfont\char237}}1
             {о}{{\selectfont\char238}}1
             {п}{{\selectfont\char239}}1
             {р}{{\selectfont\char240}}1
             {с}{{\selectfont\char241}}1
             {т}{{\selectfont\char242}}1
             {у}{{\selectfont\char243}}1
             {ф}{{\selectfont\char244}}1
             {х}{{\selectfont\char245}}1
             {ц}{{\selectfont\char246}}1
             {ч}{{\selectfont\char247}}1
             {ш}{{\selectfont\char248}}1
             {щ}{{\selectfont\char249}}1
             {ъ}{{\selectfont\char250}}1
             {ы}{{\selectfont\char251}}1
             {ь}{{\selectfont\char252}}1
             {э}{{\selectfont\char253}}1
             {ю}{{\selectfont\char254}}1
             {я}{{\selectfont\char255}}1
             {А}{{\selectfont\char192}}1
             {Б}{{\selectfont\char193}}1
             {В}{{\selectfont\char194}}1
             {Г}{{\selectfont\char195}}1
             {Д}{{\selectfont\char196}}1
             {Е}{{\selectfont\char197}}1
             {Ё}{{\"E}}1
             {Ж}{{\selectfont\char198}}1
             {З}{{\selectfont\char199}}1
             {И}{{\selectfont\char200}}1
             {Й}{{\selectfont\char201}}1
             {К}{{\selectfont\char202}}1
             {Л}{{\selectfont\char203}}1
             {М}{{\selectfont\char204}}1
             {Н}{{\selectfont\char205}}1
             {О}{{\selectfont\char206}}1
             {П}{{\selectfont\char207}}1
             {Р}{{\selectfont\char208}}1
             {С}{{\selectfont\char209}}1
             {Т}{{\selectfont\char210}}1
             {У}{{\selectfont\char211}}1
             {Ф}{{\selectfont\char212}}1
             {Х}{{\selectfont\char213}}1
             {Ц}{{\selectfont\char214}}1
             {Ч}{{\selectfont\char215}}1
             {Ш}{{\selectfont\char216}}1
             {Щ}{{\selectfont\char217}}1
             {Ъ}{{\selectfont\char218}}1
             {Ы}{{\selectfont\char219}}1
             {Ь}{{\selectfont\char220}}1
             {Э}{{\selectfont\char221}}1
             {Ю}{{\selectfont\char222}}1
             {Я}{{\selectfont\char223}}1,
    backgroundcolor=\color{backcolour},
    commentstyle=\color{codegreen}\itshape,
    keywordstyle=\color{codeblue}\bfseries,
    numberstyle=\tiny\color{codegray},
    stringstyle=\color{codegray},
    basicstyle=\ttfamily\small,
    breakatwhitespace=false,
    breaklines=true,
    captionpos=b,
    keepspaces=true,
    numbers=left,
    numbersep=5pt,
    showspaces=false,
    showstringspaces=false,
    showtabs=false,
    tabsize=4
}

% Блоки для выделения речи и подсказок
\newtcolorbox{speechbox}[1]{colback=blue!5!white, colframe=blue!40!black, fonttitle=\bfseries, title=#1}
\newtcolorbox{actionbox}[1]{colback=green!5!white, colframe=green!40!black, fonttitle=\bfseries, title=#1}
\newtcolorbox{qabox}[1]{colback=red!5!white, colframe=red!40!black, fonttitle=\bfseries, title=#1}

\begin{document}

% ===================================================================
% ТИТУЛЬНАЯ СТРАНИЦА
% ===================================================================
\begin{titlepage}
\centering

{\Large Национальный исследовательский Нижегородский государственный университет имени Н. И. Лобачевского}
\vspace{0.5cm}
(Институт информационных технологий, математики и механики)
\vspace{1cm}
{\large Кафедра: Алгебра, геометрия и дискретная математика (АГДМ)}

\vspace{2cm}

{\Large \textbf{Материалы для защиты лабораторной работы}}
\vspace{1cm}
{\large «Оптимизация последовательности переработки партий сахарной свёклы»}

\vspace{3cm}

\begin{flushright}
Выполнили:\\
Студенты группы 3823Б1ФИ1\\
Гусев Д.А.\\
Дорофеев И.Д.\\
Федосеев С.Н.\\
Никитин А.А.\\
Савва Д.А.\\

\vspace{1cm}

Научный руководитель:\\
Эгамов А.И.
\end{flushright}

\vspace{2cm}
\centering
2025 г.
\end{titlepage}

\tableofcontents
\newpage

% ===================================================================
% РАЗДЕЛ 1: ОБЩАЯ СТРУКТУРА
% ===================================================================
\section{Часть 1. Общая структура доклада}
\textit{Этот раздел содержит краткий план выступления, основные тезисы и логику построения защиты.}

\subsection{1. Введение и постановка задачи}
\begin{itemize}
    \item \textbf{Цель работы:} Разработка и исследование системы поддержки принятия решений для оптимизации технологического процесса на сахарном заводе. Главная задача — найти такую последовательность переработки партий сахарной свёклы, которая максимизирует итоговый выход сахара.
    \item \textbf{Проблема:} Процесс не является тривиальной задачей о назначениях. Сахаристость каждой партии сырья изменяется с течением времени (деградирует или дозревает), причём на каждом этапе переработки известна только текущая информация о состоянии партий. Это делает задачу «нечёткой».
\end{itemize}

\subsection{2. Математическая модель}
\begin{itemize}
    \item \textbf{Матрица сахаристости ($C$):} Описать, как формируется матрица $C$, где $c_{ij}$ — это содержание сахара в партии $i$ на этапе $j$.
    \begin{itemize}
        \item Упомянуть начальную сахаристость $a_i$ и коэффициенты деградации/дозревания $b_{ij}$.
        \item Кратко описать два вида распределения коэффициентов: \textbf{равномерное} и \textbf{концентрированное}.
    \end{itemize}
    \item \textbf{Матрица потерь ($L$):} Объяснить, что существуют неизбежные потери сахара, которые зависят от химического состава сырья ($K, Na, N$) и времени хранения. Привести формулу для расчёта $l_{ij}$.
    \item \textbf{Итоговая матрица ($\widetilde{S}$):} Представить итоговую матрицу $\widetilde{S} = C - L/100$ как основу для оптимизации.
    \item \textbf{Целевая функция:} Представить формулу для общего выхода сахара $S(\sigma) = \sum \widetilde{s}_{\sigma(j),j}$, где $\sigma$ — искомая перестановка (стратегия).
\end{itemize}

\subsection{3. Архитектура программного комплекса}
\begin{itemize}
    \item \textbf{Общая структура:} Desktop-приложение, состоящее из двух частей:
    \begin{itemize}
        \item \textbf{Frontend:} Графический интерфейс на \textbf{Electron} (HTML/CSS/JS).
        \item \textbf{Backend:} Сервер на \textbf{Python} с использованием фреймворка \textbf{Flask}.
    \end{itemize}
    \item \textbf{Схема взаимодействия:} Frontend отправляет HTTP-запросы (REST API) на Backend, получает JSON с результатами и отображает их.
\end{itemize}

\subsection{4. Реализованные алгоритмы оптимизации}
\begin{itemize}
    \item \textbf{Точный метод (Эталон):}
    \begin{itemize}
        \item \textbf{Венгерский алгоритм:} Используется как «читерский» алгоритм (зная всю матрицу $\widetilde{S}$ заранее) для получения максимально возможного выхода сахара $S^*$.
    \end{itemize}
    \item \textbf{Эвристические методы:}
    \begin{itemize}
        \item \textbf{Жадная стратегия (Greedy):} На каждом шаге $j$ выбираем партию с максимальной текущей сахаристостью.
        \item \textbf{Бережливая стратегия (Thrifty):} На каждом шаге выбираем партию с \textit{минимальной} сахаристостью.
        \item \textbf{Комбинированные стратегии:} Thrifty/Greedy, Greedy/Thrifty.
        \item \textbf{Другие эвристики:} T(k)G, Gk.
    \end{itemize}
\end{itemize}

\subsection{5. Демонстрация работы приложения}
\begin{itemize}
    \item Интерфейс и ввод параметров ($n$, $a_i$, $b_{ij}$, дозаривание).
    \item Запуск симуляции и анализ результатов (сравнение $S(\sigma)$ с эталоном).
\end{itemize}

\subsection{6. Заключение}
\begin{itemize}
    \item Выводы о том, что выбор лучшей эвристики зависит от входных параметров.
    \item Ответы на вопросы комиссии.
\end{itemize}

\newpage

% ===================================================================
% РАЗДЕЛ 2: ПОДРОБНЫЙ СЦЕНАРИЙ
% ===================================================================
\section{Часть 2. Подробный сценарий выступления}
\textit{Здесь представлен текст для слайдов и примерная речь докладчика.}

\subsection{Слайд 1: Титульный лист}
\begin{itemize}
    \item \textbf{Название:} Оптимизация последовательности переработки партий сахарной свёклы в рамках системы поддержки принятия решений.
    \item \textbf{Докладчик, руководитель, университет.}
\end{itemize}

\subsection{Слайд 2: Постановка задачи и актуальность}
\begin{speechbox}{Речь докладчика}
Целью работы является разработка программного комплекса для поиска оптимальной последовательности переработки сырья на сахарном заводе. Актуальность задачи обусловлена тем, что максимизация выхода сахара напрямую влияет на экономическую эффективность. Ключевая проблема заключается в динамическом изменении свойств сырья: решение, выгодное сейчас, может привести к потерям в будущем.
\end{speechbox}

\subsection{Слайд 3: Математическая модель}
\begin{itemize}
    \item \textbf{Входные данные:} $n$ партий свеклы, каждая с уникальными характеристиками.
    \item \textbf{Динамика:} Сахаристость $c_{ij}$ зависит от начального значения и коэффициентов $b_{ij}$.
    \begin{itemize}
        \item \textit{Увядание:} $b_{ij} < 1$.
        \item \textit{Дозаривание:} $b_{ij} > 1$.
    \end{itemize}
    \item \textbf{Целевая функция:} $S(\sigma) = \sum \widetilde{s}_{\sigma(j),j} \to \max$.
\end{itemize}

\subsection{Слайд 4: Архитектура программного комплекса}
\begin{itemize}
    \item \textbf{Диаграмма:} $[Frontend: Electron] \Leftrightarrow [REST API] \Leftrightarrow [Backend: Python/Flask]$.
    \item \textbf{Frontend:} HTML/JS/Electron — интерфейс и визуализация.
    \item \textbf{Backend:} Python/NumPy/SciPy — генерация матриц и алгоритмы оптимизации.
\end{itemize}

\subsection{Слайд 5: Алгоритмы оптимизации — Эвристики}
\begin{speechbox}{Речь докладчика}
Мы реализовали несколько эвристических стратегий:
\begin{itemize}
    \item \textbf{Жадная:} «Бери лучшее сейчас». Быстро, но часто недальновидно.
    \item \textbf{Бережливая:} «Отложи лучшее на потом». Выгодна при медленной деградации и большом разбросе качества.
    \item \textbf{Гибридные:} Сочетание подходов (например, сначала бережливая, потом жадная) часто дает наиболее робастный результат.
\end{itemize}
\end{speechbox}

\subsection{Слайд 6: Точный метод (Венгерский алгоритм)}
\begin{speechbox}{Речь докладчика}
Венгерский алгоритм используется как эталон. В реальности его применить нельзя, так как он требует знания будущего (всей матрицы $\widetilde{S}$), но он позволяет нам оценить, насколько эвристики близки к теоретическому максимуму.
\end{speechbox}

\newpage

% ===================================================================
% РАЗДЕЛ 3: СЦЕНАРИЙ ДЕМОНСТРАЦИИ
% ===================================================================
\section{Часть 3. Сценарий демонстрации}
\textit{Пошаговая инструкция для показа работы программы.}

\subsection{Шаг 1: Знакомство с интерфейсом}
\begin{actionbox}{Действие}
Открыть приложение. Обвести курсором панель с параметрами.
\end{actionbox}
\begin{speechbox}{Речь}
Перед вами главный экран системы. Слева расположена панель управления экспериментом, справа — область для вывода результатов.
\end{speechbox}

\subsection{Шаг 2: Настройка эксперимента}
\begin{actionbox}{Действие}
\begin{itemize}
    \item Количество партий ($n$): установить \textbf{10}.
    \item Начальная сахаристость ($a$): диапазон \textbf{12–18}.
    \item Коэффициент увядания ($b$): \textbf{0.98–0.99} (медленное увядание).
    \item \textbf{Включить Дозаривание}. Этапы ($v$): \textbf{3}. Коэффициент: \textbf{1.01–1.02}.
    \item \textbf{Включить Потери}.
\end{itemize}
\end{actionbox}
\begin{speechbox}{Речь}
Смоделируем ситуацию, которая хорошо покажет разницу между стратегиями. Возьмем 10 партий, включим дозаривание и зададим широкий диапазон сахаристости, чтобы были и «хорошие», и «плохие» партии.
\end{speechbox}

\subsection{Шаг 3: Запуск и анализ}
\begin{actionbox}{Действие}
Нажать кнопку «Вычислить». Дождаться таблицы.
\end{actionbox}
\begin{speechbox}{Речь}
Запускаем симуляцию.
\begin{itemize}
    \item Обратите внимание на результат \textbf{Венгерского алгоритма}. Это наш теоретический максимум $S^*$.
    \item \textbf{Жадная стратегия} показала неплохой результат, но быстро «сожгла» лучшие партии.
    \item \textbf{Бережливая стратегия} здесь, вероятно, хуже, так как при дозаривании нет смысла откладывать лучшие партии.
    \item \textbf{Гибридная стратегия Thrifty/Greedy} показывает результат, близкий к оптимальному, демонстрируя эффективность сбалансированного подхода.
\end{itemize}
\end{speechbox}

\subsection{Шаг 4: Вывод}
\begin{speechbox}{Речь}
Таким образом, программа позволяет провести вычислительный эксперимент и выбрать лучшую стратегию для конкретных производственных условий.
\end{speechbox}

\newpage

% ===================================================================
% РАЗДЕЛ 4: ВОПРОСЫ И ОТВЕТЫ
% ===================================================================
\section{Часть 4. Каверзные вопросы и ответы}

\begin{qabox}{Вопрос 1: Стохастичность}
\textbf{Вопрос:} Ваша модель потерь — детерминированная. В реальности же потери случайны. Как бы это изменило задачу?

\textbf{Ответ:} Это верное замечание. Введение стохастики усложняет модель.
\begin{enumerate}
    \item Простые эвристики стали бы менее надежными. Венгерский алгоритм в текущем виде стал бы неприменим.
    \item Для решения потребовались бы методы стохастического программирования или обучение с подкреплением (Reinforcement Learning).
    \item Для оценки стратегий пришлось бы использовать метод Монте-Карло (многократные прогоны).
\end{enumerate}
\end{qabox}

\begin{qabox}{Вопрос 2: Когда лучше «Бережливая»?}
\textbf{Вопрос:} В каком сценарии «Бережливая» стратегия будет лучшей? Она кажется контринтуитивной.

\textbf{Ответ:} Она выигрывает при совпадении трёх условий:
\begin{enumerate}
    \item \textbf{Высокий разброс качества:} есть явные «чемпионы» и «аутсайдеры».
    \item \textbf{Очень медленное увядание:} качество падает незначительно.
    \item \textbf{Отсутствие дозаривания.}
\end{enumerate}
В этом случае «жадный» потратит чемпионов вначале, когда и аутсайдеры неплохи. А «бережливый» сохранит чемпионов до конца, где их качество даст максимальный отрыв от деградировавших остальных.
\end{qabox}

\begin{qabox}{Вопрос 3: Масштабируемость}
\textbf{Вопрос:} Венгерский алгоритм имеет сложность $O(n^3)$. Что если будет 1000 партий?

\textbf{Ответ:}
\begin{enumerate}
    \item Венгерский алгоритм используется только как эталон для исследования, а не для оперативной работы. Его можно запустить один раз на мощном сервере.
    \item В качестве эталона можно использовать более быстрые аппроксимационные алгоритмы.
    \item Основной упор в работе сделан на эвристики, которые работают быстро.
\end{enumerate}
\end{qabox}

\begin{qabox}{Вопрос 4: Внедрение на заводе}
\textbf{Вопрос:} Назовите главные препятствия при внедрении системы на реальном заводе.

\textbf{Ответ:}
\begin{enumerate}
    \item \textbf{Сбор данных:} Сложно измерять сахаристость каждой партии в реальном времени.
    \item \textbf{Калибровка модели:} Нужно настроить коэффициенты $b_{ij}$ и формулу потерь под конкретный завод на основе исторических данных.
    \item \textbf{Динамика:} В реальности партии прибывают постоянно (конвейер), а наша модель статична ($n$ партий на старте). Нужно переходить к динамическому планированию («скользящее окно»).
\end{enumerate}
\end{qabox}

\begin{qabox}{Вопрос 5: Интерфейс для директора}
\textbf{Вопрос:} Интерфейс перегружен цифрами. Как сделать его понятным для директора?

\textbf{Ответ:}
\begin{enumerate}
    \item \textbf{Визуализация:} Использовать столбчатые диаграммы сравнения итоговой массы сахара вместо таблиц.
    \item \textbf{«Светофор»:} Панель рекомендаций: «Лучшая стратегия: Гибридная. Выигрыш: +3.5\%».
    \item \textbf{Интерактивность:} Возможность проиграть симуляцию по шагам.
\end{enumerate}
\end{qabox}

\end{document}